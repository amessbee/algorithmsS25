% -*- program: xelatex -*-

\newif\ifanswers
%\answerstrue

% Dr Driver's standard reading quiz
% Source: https://gist.github.com/danieldriver/90a73c4d3c72dd837e39#file-quiz-tex 
\documentclass[%
% 11pt,
% answers,
addpoints]{exam}
\usepackage{enumitem}
 % Add to the preamble section (before \begin{document})
\usepackage{tkz-graph}
\pagestyle{head}

\usepackage{amssymb,amsmath,amsfonts,amsthm,graphicx}

\usepackage[top=0.3in, bottom=0.75in, left=0.5in, right=0.5in]{geometry}
\usepackage{amsmath,amssymb}
\usepackage{tikz}


\DeclareGraphicsExtensions{.pdf}
%\centerline {\includegraphics[width=3in]{PICTURE}}



\newtheoremstyle{problem}{\topsep}{\topsep}%%% space between body and thm
		{}                      %%% Thm body font
		{}                              %%% Indent amount (empty = no indent)
		{\bfseries}            %%% Thm head font
		{}                    %%% Punctuation after thm head
		{ }                           %%% Space after thm head
		{\thmnumber{#2}\thmnote{ \bfseries (#3)}}%%% Thm head spec
\theoremstyle{problem}
\newtheorem{p}{}

\firstpageheader{% left
	CS \liningnums{3250}: Algorithms --- Asymptotic Practice Problems \\
	Husnain Haider, 4 September 2023
}{% center - blank
}{% right
	Full name1:\enspace\makebox[2in]{\hrulefill}\\
	Full name2:\enspace\makebox[2in]{\hrulefill}\\
}
\runningheader{}{}{}
% For double-sided quizzes
% \pagestyle{headandfoot}
% \footer{}{\thepage}{}


\usepackage{algorithm}
\usepackage{algorithmicx}
\usepackage{algpseudocode}
\usepackage{amsmath}
 
% Typography and layout
\usepackage{fontspec,realscripts}
\defaultfontfeatures{Ligatures=TeX}
%\setmainfont{Meta Serif Pro}
%\setsansfont{Meta Pro}
\frenchspacing
% - print solutions in sans serif
\unframedsolutions
\SolutionEmphasis{\sffamily}
\renewcommand{\solutiontitle}{}
% - box points & center in the right margin w/ custom setup@point@toks
\boxedpoints
\pointsinrightmargin
\marginbonuspointname{\textsc{up}}
\makeatletter% rewrite setup@point@toks assuming right margins
\def\clap#1{\hbox to 0pt{\hss#1\hss}}% define \clap as per https://www.tug.org/TUGboat/tb22-4/tb72perlS.pdf
\def\setup@point@toks{%
	\point@toks={%
		\rlap{\hskip-\@totalleftmargin
			  \hskip\textwidth
			  \hskip\@rightmargin
			  \hskip-\rightpointsmargin
			  \clap{\padded@point@block}% change \llap to \clap
		}%
		\global \point@toks={}%
	}%
}% end setup@point@toks
\setlength{\rightpointsmargin}{.5in}% assuming the default 1" margins
\makeatother
% - adjust the top and bottom margins
\extraheadheight{.25in}
\extrafootheight{-.5in}
\setlength{\marginparwidth}{1.5in}
% NB: remember to use \newpage after the last question
\usepackage{xcolor}
\usepackage{pagecolor}

%\pagecolor{black}
%\color{white}

\usepackage[document]{ragged2e}
\begin{document}

 \pagestyle{empty}
 \begin{FlushLeft}
Midterm Exam: Analysis of Algorithms\\Dr. Mudassir Shabbir,
\today
\end{FlushLeft}

	
\thispagestyle{myheadings}
\pagenumbering{gobble}
\rule{550pt}{1.5pt}

The exam consists of a total of 100 points, and you have 180 minutes. Make sure to write your roll-number and name on the question paper AND answersheet. 
\textbf{The solution to every question must start on a new page and complete answer to each question (all parts!) on a single page. 
Anything written beyond first ten pages of your answersheet will not be graded!}

\rule{550pt}{1.5pt}

\begin{p}
    {\bf Priority Queues}
    \begin{enumerate}
        \item Give an \( O(n \log k) \)-time algorithm to merge \( k \) sorted lists into one sorted list, where \( n \) is the total number of elements in all the input lists. (Hint: Use a min-heap for \( k \)-way merging.)
        \item Where in a max-heap might the smallest element reside, assuming that all elements are distinct?
    \end{enumerate}
    \hfill
\end{p}

\begin{p}
    {\bf Union-Find}
    \begin{enumerate}
        \item Prove that the height of a union-find data structure with path compression is at most $\log n$, i.e., $H(UF) \leq \log n$.  
        \item Consider the following sequence of Union-Find operations on elements \{1,2,3,4,5,6\}. Show the forest structure after each operation, assuming both path compression and union by rank are used. In the start, all elements are in their own set.
    \begin{enumerate}
        \item Union(1,2)
        \item Union(3,4)
        \item Union(5,6)
        \item Find(4)
        \item Union(2,3)
        \item Find(6)
        \item Union(1,5)
        \item Find(6)
    \end{enumerate}
\end{enumerate}  
\hfill
\end{p}

\begin{p}
    Let us consider the random experiment of rolling two dice.

    \begin{enumerate}
        \item Define a random variable \(X\) as the number of 6's that you get. Find the expected value of \(X\).
        \item Define a random variable \(Y\) as the sum of the two dice. Find the expected value of \(Y\).
\end{enumerate}
\hfill 
\end{p}


\begin{p}
    {\bf Articulation Points}
\begin{enumerate}
    \item Similar to articulation points, a \emph{bridge} (or cut edge) is an edge whose removal increases the number of connected components. 
    Provide an \(O(V + E)\) algorithm to identify all bridges in an undirected graph. Show how the DFS tree can be used to detect these edges.
    \item For the following undirected graph on eight vertices, find all articulation points using the algorithm discussed in the class. Show the
    discovery time and low value for the whole graph each time an articulation point is found.
    \begin{figure}[!ht]
        \centering
        \begin{tikzpicture}
            \GraphInit[vstyle=Normal]
            \SetGraphUnit{2}
            \Vertices{circle}{A,B,C,D,E,F,G,H,I}
            \Edges(A,B,C,D,E,F,G,H,A)
            \Edges(A,C)
            \Edges(B,D)
            \Edges(C,D)
            \Edges(D,E)
            \Edges(E,F)
            \Edges(F,G)
            \Edges(G,H)
            \Edges(H,E)
            \Edges(E,I)
        \end{tikzpicture}
    \end{figure}
\end{enumerate}
       
\hfill  
\end{p}


\begin{p}
    Suppose you are given a ``black box'' linear-time median finding algorithm. Design a simple linear-time algorithm to find $k^{th}$ smallest element using this algorithm. Note that median finding algorithm that you are given is black-box and you can't change it in anyway; you can just call it to find median of a list in linear-time.
\hfill 
\end{p}

\begin{p} 
    How can the number of strongly connected components of a graph on $n$ vertices change if a new edge is added? Give the extreme cases where it changes 
    drastically.
\hfill  
\end{p}
    

\begin{p} 
    Counting sort can also work efficiently if the input values have fractional parts, but the number of digits in the fractional part is small. Suppose that you are given \( n \) numbers in the range \( 0 \) to \( k \), each with at most \( d \) decimal (base 10) digits to the right of the decimal point. Modify counting sort to run in \( \Theta(n + 10^d k) \) time.  
\hfill  
\end{p}


\begin{p}
    You are given a \textbf{dataset containing 500GB} of unsorted \textbf{transaction records}, but your system has only \textbf{8GB of RAM} available for processing. Since the dataset is too large to fit into memory at once, a straightforward in-memory sorting approach is not feasible. Devise a plan to \textbf{efficiently sort} this dataset while minimizing I/O operations.
\hfill
\end{p}


\begin{p}
    The Strassen matrix multiplication algorithm follows the recurrence \(T(n) = 7T(n/2) + O(n^2)\). Use the Master Theorem to determine the time complexity of this algorithm.  
\hfill
\end{p}

\begin{p}
    Give runtime complexity of the following algorithm.
    \begin{algorithm}[!h]
        \label{algo1}
        \begin{algorithmic}
            \Procedure{Algo1}{$x$} \Comment{input $x$ is a number}
            \If{$|x| > 2$}
            \State \textbf{return} ALGO1($\sqrt{x}$)
            \Else
            \State \textbf{return} $x$
            \EndIf
            \EndProcedure
        \end{algorithmic}
    \end{algorithm}
\hfill 
\end{p}



\end{document}

% -*- program: xelatex -*-

\newif\ifanswers
%\answerstrue

% Dr Driver's standard reading quiz
% Source: https://gist.github.com/danieldriver/90a73c4d3c72dd837e39#file-quiz-tex 
\documentclass[%
% 11pt,
% answers,
addpoints]{exam}
\usepackage{enumitem}
 
\pagestyle{head}

\usepackage{amssymb,amsmath,amsfonts,amsthm,graphicx}

\usepackage[top=0.3in, bottom=0.75in, left=0.5in, right=0.5in]{geometry}
\usepackage{amsmath,amssymb}
\usepackage{tikz}


\DeclareGraphicsExtensions{.pdf}
%\centerline {\includegraphics[width=3in]{PICTURE}}



\newtheoremstyle{problem}{\topsep}{\topsep}%%% space between body and thm
		{}                      %%% Thm body font
		{}                              %%% Indent amount (empty = no indent)
		{\bfseries}            %%% Thm head font
		{}                    %%% Punctuation after thm head
		{ }                           %%% Space after thm head
		{\thmnumber{#2}\thmnote{ \bfseries (#3)}}%%% Thm head spec
\theoremstyle{problem}
\newtheorem{p}{}

\firstpageheader{% left
	CS \liningnums{3250}: Algorithms --- Finals Question Bank \\
	\today
}{% center - blank
}{% right
	Full name1:\enspace\makebox[2in]{\hrulefill}\\
	Full name2:\enspace\makebox[2in]{\hrulefill}\\
}
\runningheader{}{}{}
% For double-sided quizzes
% \pagestyle{headandfoot}
% \footer{}{\thepage}{}


\usepackage{algorithm}
\usepackage{algorithmicx}
\usepackage{algpseudocode}
\usepackage{amsmath}
 
% Typography and layout
\usepackage{fontspec,realscripts}
\defaultfontfeatures{Ligatures=TeX}
%\setmainfont{Meta Serif Pro}
%\setsansfont{Meta Pro}
\frenchspacing
% - print solutions in sans serif
\unframedsolutions
\SolutionEmphasis{\sffamily}
\renewcommand{\solutiontitle}{}
% - box points & center in the right margin w/ custom setup@point@toks
\boxedpoints
\pointsinrightmargin
\marginbonuspointname{\textsc{up}}
\makeatletter% rewrite setup@point@toks assuming right margins
\def\clap#1{\hbox to 0pt{\hss#1\hss}}% define \clap as per https://www.tug.org/TUGboat/tb22-4/tb72perlS.pdf
\def\setup@point@toks{%
	\point@toks={%
		\rlap{\hskip-\@totalleftmargin
			  \hskip\textwidth
			  \hskip\@rightmargin
			  \hskip-\rightpointsmargin
			  \clap{\padded@point@block}% change \llap to \clap
		}%
		\global \point@toks={}%
	}%
}% end setup@point@toks
\setlength{\rightpointsmargin}{.5in}% assuming the default 1" margins
\makeatother
% - adjust the top and bottom margins
\extraheadheight{.25in}
\extrafootheight{-.5in}
\setlength{\marginparwidth}{1.5in}
% NB: remember to use \newpage after the last question
\usepackage{xcolor}
\usepackage{pagecolor}

%\pagecolor{black}
%\color{white}

\usepackage[document]{ragged2e}
\begin{document}

 \pagestyle{empty}
 \begin{FlushLeft}
Finals Question Bank: Analysis of Algorithms\\
\today
\end{FlushLeft}

	
\thispagestyle{myheadings}
\pagenumbering{gobble}
\rule{500pt}{1.5pt}

% Questions from q1.tex
\begin{p}
Let us consider the random experiment of rolling two dice.
\begin{enumerate}
    \item Define a random variable \(X\) as the number of 6's that you get. Find the expected value of \(X\).
    \item Define a random variable \(Y\) as the sum of the two dice. Find the expected value of \(Y\).
\end{enumerate}
\hfill \end{p}

\begin{p}
Perform the analysis of Median of Median's Algorithm where we make chunks of size 9 instead of 5 and solve the recurrence using substitution method.
\hfill \end{p}

% Questions from q2.tex
\begin{p}
Let \( X \) be a random variable that is equal to the number of heads in two flips of a  
fair coin. What is \( E(X^2) \)? What is \( (E(X))^2 \)?
\hfill \end{p}

% Questions from q3.tex
\begin{p}
Apply the algorithm step by step to find the articulation points in the following graphs:
\end{p}

\begin{center}
\begin{tikzpicture}[scale=1, every node/.style={draw, circle, inner sep=2pt}]
    \node (1) at (0, 2) {1};
    \node (2) at (2, 2) {2};
    \node (3) at (4, 2) {3};
    \node (4) at (1, 0) {4};

    \draw (1) -- (2);
    \draw (2) -- (3);
    \draw (4) -- (2);
    \draw (1) -- (4);
\end{tikzpicture}
\end{center}

\begin{center}
\begin{tikzpicture}[scale=1, every node/.style={draw, circle, inner sep=2pt}]
    \node (1) at (0, 2) {1};
    \node (2) at (2, 2) {2};
    \node (3) at (4, 2) {3};
    \node (4) at (1, 0) {4};
    \node (5) at (3, 0) {5};
    \node (6) at (5, 0) {6};

    \draw (1) -- (2);
    \draw (2) -- (3);
    \draw (2) -- (4);
    \draw (3) -- (5);
    \draw (5) -- (6);
\end{tikzpicture}
\end{center}

% Questions from q4.tex
\begin{p}
Given an undirected graph with distinct non-negative edge weights. Suppose that you have computed shortest paths to all nodes from a particular node \( s \). Now suppose each edge weight is increased by 1; the new weights are \( w_e := w_e + 1 \). Do the shortest paths change? Give an example where they change or prove they cannot change.
\hfill \end{p}

\begin{p} 
Give a general example of a weighted, directed graph \( G \) on \( n \) nodes where at least half of the edge weights are negative, and Dijkstra's algorithm correctly finds the shortest paths from a source vertex. Note that \( n \) is not fixed, so you cannot just give an example on 10 or 20 vertices. Your job is to maximize the number of edges that contain negative weights.
\hfill \end{p}

% Questions from q5.tex
\begin{p}
How can we use the output of the Floyd-Warshall algorithm to detect the presence of a negative-weight cycle?
\hfill \end{p}

% Questions from q6.tex
\begin{p}
"In GoldExtractor game, you collect coins using two extractors: a red vertical one that moves right and a blue horizontal one that moves up. Each time an extractor moves to a new line containing coins, it collects one coin, and all other coins on that line are lost. The game ends when either extractor reaches its final position, so plan your moves carefully to maximize your total score!"

Design a complete Dynamic Programming algorithm for the problem. Your input is (x,y) coordinates of n coins, and your output is a number which is maximum number of coins collected. You can assume that $0 < x,y < n$.
\hfill \end{p}

% Questions from q7.tex
\begin{p}
Find the decision version of the minimum spanning tree problem. State the original problem and the decision problem formally.
\hfill \end{p}

\begin{p} 
\textbf{Rod Cutting Problem}\\
Given a rod of length $n$ and an array of prices \texttt{price} such that \texttt{price[i]} represents the price of a rod of length $i+1$, determine the maximum value obtainable by cutting up the rod and selling the pieces.
\begin{itemize}
  \item \textbf{Input:} $n = 8$, \texttt{price} = [1, 5, 8, 9, 10, 17, 17, 20]
  \item \textbf{Output:} 22 (Cut into rods of lengths 2 and 6 for prices 5 and 17)
\end{itemize}
\hfill \end{p}

% Questions from q8.tex
\begin{p}
Prove that the following problem is in NP:\\ 
\hspace{9pt}Given an integer $x$, is $x$ NOT a prime?
\hfill \end{p}

\begin{p} 
Reduce 4-SAT to 3-SAT.
\hfill \end{p}

% Questions from q9.tex
\begin{p}
Reduce Subset Sum to Partition problem.
\hfill \end{p}

\begin{p} 
Reduce Partition to Subset Sum problem.
\hfill \end{p}

% Questions from q10.tex
\begin{p}
Reduce Partition to Bin Packing probem.
\hfill \end{p}

% Questions from q11.tex
\begin{p}
Reduce Vertex Cover to Set Cover problem.
\hfill \end{p}

% Questions from q12.tex
\begin{p}
Reduce Maximum Clique to Subgraph Isomorphism problem.
\hfill \end{p}

% Questions from q13.tex
\begin{p}
Reduce Subset Sum to Knapsack problem.
\hfill \end{p}

% Questions from q14.tex
\begin{p}
Reduce Vertex Cover to Maximum Dominating Set problem.
\hfill \end{p}

% Questions from w1.tex
\begin{p}
Solve the following recurrence relations:\\
\vspace{10pt}
\textbf{a.} \( T(n) = 2T(n/3) + 1 \)\\
\textbf{b.} \( T(n) = 5T(n/4) + n \)\\
\textbf{c.} \( T(n) = 9T(n/3) + n^2 \)\\
\textbf{d.} \( T(n) = 2T(n-1) + 1 \)\\
\textbf{e.} \( T(n) = T(\sqrt{n}) + 1 \)\\
\hfill \end{p}

\begin{p}
An array \( A[1 \dots n] \) is said to have a \textbf{majority element} if more than half of its entries are the same. Given an array, the task is to design an efficient algorithm to determine whether the array has a majority element, and, if so, to find that element. The elements of the array are not necessarily from some ordered domain like the integers, so comparisons of the form "is \( A[i] > A[j]? \)" are not allowed. (Think of the array elements as GIF files, for instance.) However, you can answer questions of the form: "is \( A[i] = A[j]? \)" in constant time.\\
\vspace{10pt}
\textbf{Part 1:} Show how to solve this problem in \( O(n \log n) \) time. 
\textbf{Hint:} Split the array \( A \) into two arrays \( A_1 \) and \( A_2 \) of half the size. Does knowing the majority elements of \( A_1 \) and \( A_2 \) help you figure out the majority element of \( A \)? If so, you can use a \textit{divide-and-conquer} approach.\\
\vspace{10pt}
\textbf{Part 2:} Can you give a linear-time algorithm? 
\textbf{Hint:} Here's another \textit{divide-and-conquer} approach:
\begin{itemize}
    \item Pair up the elements of \( A \) arbitrarily to get \( n/2 \) pairs.
    \item Look at each pair: if the two elements are different, discard both of them; if they are the same, keep just one of them.
\end{itemize}
Show that after this procedure there are at most \( n/2 \) elements left, and that they have a majority element if and only if \( A \) does.
\hfill \end{p}

% Questions from w2.tex
\begin{p}
Write an algorithm to merge three sorted arrays into a single sorted array.\\
\hfill \end{p}

\begin{p}
Write an algorithm to merge \( k \) sorted arrays into a single sorted array.\\
\hfill \end{p}

\begin{p}
Given two arrays \( A \) and \( B \), write an algorithm to find the median of \( A \cup B \) in the following cases:  
\hfill \end{p}
\begin{itemize}
    \item When \( A \) and \( B \) are sorted.
    \item When \( A \) and \( B \) are unsorted.
\end{itemize}

% Questions from w3.tex
\begin{p}
Give an \( O(n \log k) \)-time algorithm to merge \( k \) sorted lists into one sorted list,  
where \( n \) is the total number of elements in all the input lists. (Hint: Use a min-heap for \( k \)-way merging.)
\hfill \end{p}

\begin{p}
You are given an array of strings, where different strings may have different  
numbers of characters, but the total number of characters over all the strings  
is \( n \). Show how to sort the strings in \( O(n) \) time.  
(Note that the desired order here is the standard alphabetical order; for example,  
\( a < ab < b \).)
\hfill \end{p}

% Questions from w4.tex
\begin{p}
What is the running time of the Breadth-First Search (BFS) algorithm if we represent its input graph using an adjacency matrix and modify the algorithm to handle this form of input?
\hfill \end{p}

\begin{p}
The \textbf{diameter} of a tree \( T = (V, F) \) is defined as \( \max \{ \delta(u, v) : u, v \in V \} \), that is, the largest of all shortest-path distances in the tree. Give an efficient algorithm to compute the diameter of a tree, and analyze the running time of your algorithm.
\hfill \end{p}

% Questions from w5.tex
\begin{p}
For each of the following strings:
\begin{enumerate}[label=\alph*.]
    \item Build a frequency table for the characters.
    \item Construct the Huffman merge tree.
    \item Generate Huffman codes for each character.
    \item Encode the full string using the Huffman codes.
    \item Compare the number of bits in the Huffman-encoded version with standard ASCII encoding (8 bits per character).
\end{enumerate}

\texttt{AGCTTAGGCTTAA}
\hfill \end{p}

% Questions from w6.tex
\begin{p}
\textbf{Coin Change Problem} \\[1ex]
Given coin denominations $c_1, c_2, \ldots, c_n$ and a target amount $A$, find the minimum number of coins needed to make change for the amount $A$. Assume an infinite supply of coins of each denomination.
\hfill \end{p}

\begin{p}
\textbf{Longest Increasing Subsequence}\\[1ex]
Given a sequence of integers $A = a_1, a_2, \ldots, a_n$, find the length of the longest subsequence such that all elements of the subsequence are sorted in increasing order.
\hfill \end{p}

% Questions from w7.tex
\begin{p}
Reduce Maximum Independent Set to Maximum Clique problem.
\hfill \end{p}

% Questions from w8.tex
\begin{p}
Reduce Hamiltonian Cycle to Hamiltonian Path problem.
\hfill \end{p}

\newpage

% -------------------------------------------------------------------------
% By Muhammad Wasif
% -------------------------------------------------------------------------

\begin{p}
\textbf{Box Stacking Problem}

\textbf{Definition:} Given a set of $n$ boxes, each with height $h_i$, width $w_i$, and depth $d_i$, determine the maximum possible height of a stack that can be formed under the following constraints:
\begin{itemize}
\item A box can be rotated so that any side functions as its base.
\item A box can only be stacked on top of another if both its width and depth are strictly less than those of the box below it.
\item Only one instance of each box can be used in the stack.
\end{itemize}

\textbf{Input:} An array of $n$ boxes, each defined by three integers: height $h_i$, width $w_i$, and depth $d_i$.

\textbf{Output:} Maximum stack height achievable by stacking the boxes.

\textbf{Recurrence Relation:}
\[ \text{DP}[i] = \max (h_i, h_i + \text{DP}[j] \mid j < i \text{ and } w_i < w_j \text{ and } d_i < d_j ) \]

\textbf{Steps:}
\begin{enumerate}
\item Generate all rotations of each box (3 per box) so width $\geq$ depth.
\item Sort all $3n$ boxes by decreasing base area ($w \times d$).
\item Apply the recurrence using bottom-up DP.
\end{enumerate}

\textbf{Example:} Input: Boxes = [(4,6,7), (1,2,3), (4,5,6)]\\
Output: 15
\hfill \end{p}

\begin{p}
\textbf{3SAT to Mario Level Completion}

\textbf{Definition:}
\begin{itemize}
\item \textbf{3SAT:} Given a Boolean formula in CNF with exactly three literals per clause, determine whether there exists a truth assignment that satisfies the formula.
\item \textbf{Mario Level Completion:} Given a Mario-style platform level with switches, obstacles, and doors, determine whether Mario can reach the goal under the game's rules.
\end{itemize}

\textbf{Example (3SAT):} Formula: $(x_1 \vee \neg x_2 \vee x_3) \wedge (\neg x_1 \vee x_2 \vee x_3)$\\
Satisfying assignment: $x_1 = \text{true}$, $x_2 = \text{false}$, $x_3 = \text{true}$

\textbf{Example (Mario Level):} Each variable corresponds to a choice of path: assigning true or false blocks the other path. Each clause is a locked door opened if at least one of its literals' paths has been taken.

\textbf{Task:} Reduce from 3SAT to Mario Level Completion. Construct a Mario level such that Mario can reach the goal if and only if the 3SAT formula is satisfiable.
\hfill \end{p}

\begin{p}
\textbf{Huffman Coding}

\textbf{Definition:} Given a string of characters, generate prefix-free binary codes using Huffman's algorithm to minimize the total encoded message length.

\textbf{Input String:} "data structures and algorithms"

\textbf{Tasks:}
\begin{enumerate}
\item Count the frequency of each character (including spaces).
\item Build the Huffman tree from frequencies.
\item Assign binary codes to each character.
\item Encode the original string using those binary codes.
\end{enumerate}

\textbf{Output:} A binary string representing the Huffman encoding of the input, and the binary codes used.
\hfill \end{p}

\begin{p}
\textbf{Knapsack to Bin Partitioning}

\textbf{Definition:}
\begin{itemize}
\item \textbf{Knapsack Problem:} Given $n$ items with weights $w_i$ and values $v_i$, and capacity $W$, determine the maximum total value of a subset of items whose total weight does not exceed $W$.
\item \textbf{Bin Partitioning Problem (Decision Version):} Given $n$ items with weights $w_i$, a bin capacity $C$, and an integer $k$, determine whether the items can be partitioned into $k$ or fewer bins, such that the total weight in each bin does not exceed $C$.
\end{itemize}

\textbf{Example (Knapsack):}\\
Items = $\{(4,6), (2,4), (3,5), (5,8)\}$, $W = 7$\\
Max value = 11 (select items with weights 2 and 5)

\textbf{Example (Bin Partitioning):}\\
Item weights = $\{4, 2, 3, 5\}$, $C = 7$, $k = 2$\\
Valid bins: $\{2, 5\}$, $\{3, 4\}$

\textbf{Task:} Reduce from Knapsack to Bin Partitioning. Construct an instance of the bin problem such that solving it corresponds to solving the original Knapsack instance.
\hfill \end{p}

\begin{p}
\textbf{All-Pairs Shortest Paths Using Floyd-Warshall Algorithm}

\textbf{Problem Definition:}\\
Given a directed graph $G = (V, E)$ with edge weights (positive or negative, but no negative cycles), compute the shortest path distances between every pair of vertices using the Floyd-Warshall algorithm.

\textbf{Input:}\\
A weighted directed graph with $n = 5$ vertices and the following edges shown in the diagram:

\begin{center}
    \begin{tikzpicture}[shorten >=1pt, auto, node distance=2.8cm,
        thick, main node/.style={circle, draw, font=\sffamily\Large\bfseries},
        edge/.style={->, >=stealth, thick}]
      
        \node[main node] (0) {0};
        \node[main node] (1) [right of=0] {1};
        \node[main node] (2) [below left of=0] {2};
        \node[main node] (3) [below right of=2] {3};
        \node[main node] (4) [right of=3] {4};
      
        \path[every node/.style={font=\sffamily\small}]
          (0) edge node {4} (1)
              edge node {5} (3)
          (1) edge node {6} (4)
              edge[bend right=90] node {1} (2)
          (2) edge node {3} (3)
              edge node {2} (0)
          (3) edge node {2} (4)
              edge[bend left=40] node {1} (2)
          (4) edge[bend left=40] node {4} (3)
              edge node {1} (0)
          ;
      
      \end{tikzpicture}
\end{center}

\textbf{Output:}\\
A matrix $D$ such that $D[i][j]$ contains the shortest path distance from vertex $i$ to vertex $j$.

\textbf{Recurrence Relation:}\\
Let $D^{(k)}[i][j]$ be the shortest distance from vertex $i$ to $j$ using only intermediate vertices from the set $\{0, 1, ..., k\}$. Then:

$D^{(0)}[i][j] = 
\begin{cases} 
0 & \text{if } i = j \\
w(i, j) & \text{if } (i, j) \in E \\
\infty & \text{otherwise}
\end{cases}$

$D^{(k)}[i][j] = \min\{D^{(k-1)}[i][j], D^{(k-1)}[i][k] + D^{(k-1)}[k][j]\}$

\textbf{Example:} Using the provided graph, the initial distance matrix $D^{(0)}$ is:
$\begin{pmatrix}
0 & 4 & 2 & 5 & 1 \\
\infty & 0 & \infty & \infty & 6 \\
\infty & 1 & 0 & 3 & \infty \\
\infty & \infty & \infty & 0 & 2 \\
\infty & \infty & \infty & 4 & 0
\end{pmatrix}$

After applying Floyd-Warshall, the final distance matrix $D$ gives the shortest paths between all pairs of vertices.
\hfill \end{p}

% -------------------------------------------------------------------------
% By Jalal Ahmed
% -------------------------------------------------------------------------

\begin{p}
\textbf{Smallest Number with Given Digit Count and Sum}

Given two integers $s$ and $d$, find the smallest possible number that has exactly $d$ digits and a sum of digits equal to $s$. Return the number as a string. If no such number exists, return "-1".

\textbf{Examples:}
\begin{itemize}
\item \textbf{Input:} $s = 9$, $d = 2$\\
      \textbf{Output:} "18"\\
      \textbf{Explanation:} 18 is the smallest number possible with the sum of digits = 9 and total digits = 2.
      
\item \textbf{Input:} $s = 20$, $d = 3$\\
      \textbf{Output:} "299"\\
      \textbf{Explanation:} 299 is the smallest number possible with the sum of digits = 20 and total digits = 3.
      
\item \textbf{Input:} $s = 1$, $d = 1$\\
      \textbf{Output:} "1"\\
      \textbf{Explanation:} 1 is the smallest number possible with the sum of digits = 1 and total digits = 1.
\end{itemize}
\hfill \end{p}

\begin{p}
\textbf{Minimum Spanning Tree (MST)}

Given an undirected, weighted and connected graph $G$ with $n$ vertices and $m$ edges, find the weight of the Minimum Spanning Tree.

\textbf{Example:}
\begin{center}
\begin{tikzpicture}[scale=1.2, every node/.style={draw, circle, inner sep=2pt}]
    \node (1) at (0, 2) {1};
    \node (2) at (2, 2) {2};
    \node (3) at (0, 0) {3};
    \node (4) at (2, 0) {4};

    \draw (1) -- node[draw=none, fill=none, above] {1} (2);
    \draw (1) -- node[draw=none, fill=none, left] {4} (3);
    \draw (2) -- node[draw=none, fill=none, right] {2} (3);
    \draw (3) -- node[draw=none, fill=none, below] {3} (4);
\end{tikzpicture}
\end{center}

MST weight = 6 (edges with weights 1, 2, and 3)
\hfill \end{p}

\begin{p}
Given an array of $n$ integers, find the length of the longest increasing subsequence.

\textbf{Example:}
Given an array $A = [ 10, 9, 2, 5, 3, 7, 101, 18 ]$\\
LIS = $[ 2, 3, 7, 101 ]$ and the length is 4.
\hfill \end{p}

\begin{p}
Show how the Maximum Independent Set problem can be reduced to the Maximum Clique problem.
\hfill \end{p}

\begin{p}
\textbf{Huffman Coding}

Construct a Huffman Tree using the given frequency table, then determine the Huffman Encoding for each character.

\begin{center}
\begin{tabular}{|c|c|}
\hline
\textbf{Character} & \textbf{Frequency} \\
\hline
A & 5 \\
B & 9 \\
C & 12 \\
D & 13 \\
E & 16 \\
F & 45 \\
\hline
\end{tabular}
\end{center}
\hfill \end{p}

% -------------------------------------------------------------------------
% By Fahad
% -------------------------------------------------------------------------

\begin{p}
\textbf{Minimum Cost with K Stops (Flight Problem)}

Given flights between cities with costs, a source, destination, and max K stops allowed, find the minimum cost path using a modified Dijkstra or BFS+DP approach.
\hfill \end{p}

\begin{p}
\textbf{SSSP with Edge Penalty}

Each edge has a weight $w$ and a penalty $p$. If you take two consecutive edges with the same penalty, you pay the penalty only once. Find the shortest path from the source to destination minimizing the total cost (weight + penalty rules).
\hfill \end{p}

\begin{p}
\textbf{Transitive Closure using Floyd-Warshall}

Given a directed graph, compute the transitive closure (i.e., for every pair $(u, v)$, determine if $v$ is reachable from $u$). You must modify Floyd-Warshall to work with boolean matrix.
\hfill \end{p}

\begin{p}
\textbf{Minimum Number of Platforms (Train Problem)}

Given arrival and departure times of trains at a station, find the minimum number of platforms required so that no train waits.
\hfill \end{p}

\begin{p}
\textbf{Longest Bitonic Subsequence}

Find the length of the longest subsequence that first increases and then decreases. Combine LIS from left and LDS from right for each position.
\hfill \end{p}

\begin{p}
\textbf{Subset Product Problem}

Given a set of positive integers and a target product $P$, is there a subset whose product is exactly $P$?
\hfill \end{p}

% -------------------------------------------------------------------------
% By M Hamza Naveed
% -------------------------------------------------------------------------

\begin{p}
\textbf{Knapsack Problem}

You're going on a treasure hunt with a backpack that can carry only $W$ kg. There are $n$ items, each with a weight and value. You can take each item only once or not at all. Which items should you pick to get maximum value without overloading your backpack?
\hfill \end{p}

\begin{p}
\textbf{Fractional Knapsack}

You're still collecting treasure! But this time, your magical bag allows you to take fractions of items. You can take half a gold bar if needed. Your goal? Maximize the loot you can carry in the same $W$ kg bag.

And will greedy approach fail? Explain with example.
\hfill \end{p}

\begin{p}
\textbf{Travelling Salesman Problem (TSP)}

You are a delivery person. You must visit all cities once, then return home. Your task is to find the shortest path that does this.

You're given a weighted graph of cities and a bound of total cost you incur during your trip. For this problem, prove it is an NP-hard problem.
\hfill \end{p}

\begin{p}
\textbf{Transitive Closure using Floyd–Warshall}

You want to know who is related to whom in a network. Even if there's no direct relation, can person A be related to person B through others?

Instead of a list of all relations, you just want Yes/No for every pair.
\hfill \end{p}

\begin{p}
\textbf{Bellman–Ford Algorithm – Single Source Shortest Path}

You're navigating a graph that might have negative tolls (yep, they pay you!). Find the shortest path from one place to all others.

Also, detect if there's a nasty loop giving infinite money.
\hfill \end{p}

% -------------------------------------------------------------------------
% By Khadija Farooqi
% -------------------------------------------------------------------------

\begin{p}

Given an instance of the CLIQUE problem, construct a polynomial-time transformation that converts it into an instance of the INTERVAL SCHEDULING WITH CONSTRAINTS problem such that:

The original graph $G=(V,E)$ contains a clique of size $k$ if and only if the corresponding interval scheduling instance allows selecting $k$ non-overlapping intervals that satisfy the specified constraints.

\textbf{Definitions:}
\begin{itemize}
\item \textbf{CLIQUE Problem:} Given a graph $G=(V,E)$ and an integer $k$, does $G$ contain a clique (a complete subgraph) of size $k$?
\item \textbf{Interval Scheduling with Constraints:} Given a set of intervals with associated constraints (e.g., mutual exclusivity, resource limits, or time gaps), is it possible to select a subset of $k$ mutually compatible intervals?
\end{itemize}
\hfill \end{p}

\begin{p}

Given an instance of the Travelling Salesman Problem (TSP), construct a polynomial-time transformation that produces an instance of the Hamiltonian Cycle problem such that:

The original TSP instance has a tour of total weight $\leq K$ if and only if the corresponding graph in the HAM-CYCLE instance contains a Hamiltonian cycle.

\textbf{Definitions:}
\begin{itemize}
\item \textbf{Travelling Salesman Problem (Decision version):} Given a complete weighted graph $G=(V,E)$ and an integer $K$, does there exist a tour (a cycle visiting every vertex exactly once and returning to the start) of total weight less than or equal to $K$?
\item \textbf{Hamiltonian Cycle Problem:} Given an unweighted graph $G=(V,E)$, does there exist a cycle that visits every vertex exactly once?
\end{itemize}
\hfill \end{p}

\begin{p}
Why does not Dijkstra work for the negative weights? Explain with the reasoning.
\hfill \end{p}

\begin{p}

Given a 2D binary matrix of 0s and 1s, find the area of the largest square containing only 1s.

\textbf{Example Input:} 
\begin{verbatim}
Matrix:
1 0 1 0 0
1 0 1 1 1
1 1 1 1 1
1 0 0 1 0
\end{verbatim}

\textbf{Expected Output:} Area = 9 (3x3 square)
\hfill \end{p}

\begin{p}

You are continuously inserting integers into a list. Return the kth largest element after each insertion.
\hfill \end{p}

% -------------------------------------------------------------------------
% By Abuhurairah Faheem
% -------------------------------------------------------------------------

\begin{p}
\textbf{House Robber Problem}

You are a professional robber planning to rob houses along a street. Each house has a certain amount of money stashed, and you cannot rob two adjacent houses because it will alert the police. Given an array of non-negative integers representing the amount of money in each house, determine the maximum amount of money you can rob without alerting the police.

\textbf{Example:}\\
Input: nums [2, 7, 9, 3, 1]\\
Output: 12\\
Explanation: Rob house 1 (\$2), skip house 2 (\$7), rob house 3 (\$9), and rob house 5 (\$1) for a total of 2+9+1 = 12.
\hfill \end{p}

\begin{p}
\textbf{All-Pairs Shortest Paths (Floyd-Warshall Algorithm)}

Given a directed graph with n vertices and weighted edges (some weights may be negative, but no negative cycles), find the shortest paths between all pairs of vertices.

\textbf{Input:}\\
Graph represented as an adjacency matrix:
\begin{verbatim}
[ [0 3 ∞ ∞]
  [∞ 0 1 ∞]
  [∞ ∞ 0 2]
  [2 ∞ ∞ 0] ]
\end{verbatim}

\textbf{Output:}\\
Shortest-path matrix.
\hfill \end{p}

\begin{p}
\textbf{Dijkstra Algorithm on Graph with Negative Edge}

\begin{itemize}
\item Run Dijkstra from A to C on the graph with edges A→B(2), B→C(–1), A→C(4). What shortest path and distance does it find?
\item Does Dijkstra produce the correct shortest path here despite the negative edge?
\end{itemize}
\hfill \end{p}

\begin{p}
\textbf{Compress a File Header}

Given a list of characters and their frequencies from a text file header, build the Huffman tree and output the code for each character.

\textbf{Input:} \{a: 5, b: 9, c: 12, d: 13\}

\textbf{Task:} Build Huffman codes.
\hfill \end{p}

\begin{p}
\textbf{Reduction: Partition to Job Schedule}

Reduce Partition problem to Job schedule problem.
\hfill \end{p}

% -------------------------------------------------------------------------
% By Muhammad Ibrahim Butt
% -------------------------------------------------------------------------

\begin{p}
\textbf{3-SAT to Vertex Cover (NP-Complete Reduction)}
\textbf{What:}
A way to convert a 3-SAT logical formula into a graph so that finding a vertex cover corresponds exactly to finding a satisfying assignment for the formula.

\textbf{How:}
\begin{itemize}
\item For each variable $x_i$, create two nodes $x_i$ and $\neg x_i$ connected by an edge (representing the choice of true/false).
\item For each clause (3 literals), create a triangle of three nodes connected in a cycle.
\item Connect each literal node in the clause triangle to its corresponding variable node.
\item The vertex cover size $k = n + 2m$ where $n$ = variables and $m$ = clauses.
\end{itemize}

\textbf{Why:} Picking one vertex from each variable pair corresponds to a truth assignment. The clause triangles force the selection of vertices such that the clause is "covered" (satisfied). Thus, vertex cover solutions correspond to formula satisfiability.
\hfill \end{p}

\begin{p}
\textbf{SPFA (Shortest Path Faster Algorithm)}
\textbf{What:}
An optimized version of Bellman-Ford algorithm to find shortest paths from a single source, especially for graphs with negative edges but no negative cycles.

\textbf{How:}
\begin{itemize}
\item Uses a queue to process nodes whose distance estimate can still be improved.
\item Nodes are added to the queue only when their distance is updated, reducing unnecessary relaxations.
\item Faster on average than Bellman-Ford because it avoids processing all edges every iteration.
\end{itemize}

\textbf{When to use:} Graphs with negative weights but no negative cycles, where Dijkstra's algorithm cannot be applied.
\hfill \end{p}

\begin{p}
\textbf{Binary Search + Patience Sorting for LIS (Longest Increasing Subsequence)}
\textbf{What:}
Efficient method to find the length of the Longest Increasing Subsequence in $O(n \log n)$ time.

\textbf{How:}
\begin{itemize}
\item Maintain a list $tails$, where $tails[i]$ is the smallest possible tail value of an increasing subsequence of length $i+1$.
\item For each element $x$ in the sequence, use binary search on $tails$ to find the position to place $x$ (replace or append).
\item The length of $tails$ at the end is the length of LIS.
\end{itemize}

\textbf{Why:} Patience sorting mimics the card game and the binary search efficiently maintains the subsequence tails.
\hfill \end{p}

\begin{p}
\textbf{Johnson's Algorithm (All Pairs Shortest Path)}
\textbf{What:}
Computes shortest paths between all pairs of vertices in a weighted, directed graph which may contain negative edges but no negative cycles.

\textbf{How:}
\begin{itemize}
\item Add a new vertex connected to every other vertex with zero-weight edges.
\item Run Bellman-Ford from this new vertex to find a potential function $h(v)$ to reweight edges and remove negative weights.
\item Reweight edges using $h(u), h(v)$: $w'(u,v) = w(u,v) + h(u) - h(v)$, all non-negative now.
\item Run Dijkstra's algorithm from each vertex on the reweighted graph.
\item Correct final distances by reversing reweighting.
\end{itemize}

\textbf{Why:} Combines Bellman-Ford's ability to handle negative weights and Dijkstra's efficiency on non-negative weights.
\hfill \end{p}

\begin{p}
\textbf{Activity Selection Problem}

\textbf{What:}
Select the maximum number of activities that don't overlap given their start and finish times.

\textbf{How:}
\begin{itemize}
\item Sort activities by finish times.
\item Iteratively pick the earliest finishing activity that starts after the last selected one.
\end{itemize}

\textbf{Why:} This greedy choice ensures optimality because picking the activity that frees the schedule earliest maximizes remaining time for others.
\hfill \end{p}

\end{document} 